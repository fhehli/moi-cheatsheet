\documentclass[8pt,a4paper]{extarticle}     % Necessary to make small margin an small text paper.
\usepackage[utf8]{inputenc}	                % UTF-8 characters.
\usepackage[landscape, margin=1cm, bmargin=0.5cm, includefoot, footskip=0.5cm]{geometry}            % Necessary for landscape paper.
\usepackage[textsize=tiny]{todonotes}       % Necessary for small margins.
\usepackage{enumitem}                       % Necessary for personalized lists.
\usepackage{mdframed}                       % Necessary for dark gray boxes.
\usepackage{mathtools}                      % More compact math.
\usepackage{amsthm}                         % Necessary for theorem boxes.
\usepackage{amssymb}						% Necessary for mathbb symbols.
\usepackage{multicol,multirow}              % Necessary for multi column format.
\usepackage{subfiles}						% Necessary for multiple subfiles
\usepackage{tabularx}						% Necessary for full-column tables 
\usepackage{bm}								% Necessary for bold math with \pbm{...}
\usepackage{xcolor}							% Necessary for custom colors. 
\usepackage{graphicx}
\usepackage{accents}						% Necessary for doublehat
\usepackage{pgfplots}
\usepackage{fancyhdr}						% Necessary for header/footer settings

\newlist{listnr}{enumerate}{1}
\setlist[listnr, 1]{label={(\roman*)},itemsep=-0.2em, leftmargin=*,labelindent=-0.5em}


%% Special characters for number sets, e.g. real or complex numbers.
\newcommand{\C}{\mathbb{C}}
\newcommand{\K}{\mathbb{K}}
\newcommand{\N}{\mathbb{N}}
\newcommand{\Q}{\mathbb{Q}}
\newcommand{\R}{\mathbb{R}}
\newcommand{\Z}{\mathbb{Z}}
\newcommand{\X}{\mathbb{X}}
\renewcommand{\P}{\mathbb{P}}

\newcommand{\No}{\mathcal{N}}

\newcommand{\br}{\par\medskip\noindent}
\newcommand{\E}{\mathbb{E}}
% \newcommand{\P}{\mathbb{P}}
\newcommand{\Oh}{\mathcal{O}}
% \newcommand{\vphi}{\varphi}
% \newcommand{\veps}{\varepsilon} 
\newcommand{\bd}{\textbf}
\newcommand{\equi}{\Leftrightarrow}
\newcommand{\imp}{\Rightarrow}
\newcommand{\emp}{\varnothing}
\newcommand{\subs}{\subseteq}
\newcommand{\ol}{\overline}
\newcommand{\ra}{\rangle}
\newcommand{\la}{\langle}
\newcommand{\ox}{\otimes}
\newcommand*\dx{\mathop{}\!\mathrm{d}x}
\newcommand*\dy{\mathop{}\!\mathrm{d}y}
\newcommand*\dz{\mathop{}\!\mathrm{d}z}

\DeclareMathOperator*{\argmax}{arg\,max}
\DeclareMathOperator*{\argmin}{arg\,min}
\DeclareMathOperator*{\vol}{vol}
\DeclareMathOperator*{\Mat}{Mat}
\DeclareMathOperator*{\id}{id}
\DeclareMathOperator*{\tr}{tr}
\DeclareMathOperator*{\rank}{rank}
\DeclareMathOperator*{\im}{im}

\newcommand{\triv}[1]{{\left\vert\kern-0.25ex\left\vert\kern-0.25ex\left\vert #1 
    \right\vert\kern-0.25ex\right\vert\kern-0.25ex\right\vert}}

% Header / Footer settings
\pagestyle{fancy}
\renewcommand{\headrulewidth}{0pt}
\rhead{} 
\lhead{} 
\chead{} 

% For Figures
\usetikzlibrary{decorations.markings}
\pgfplotsset{compat=1.11}

%---------------------------------------------------%
% Environments
%---------------------------------------------------%

\mdtheorem [%
		backgroundcolor	= black!10,%
		topline			= false,%
		bottomline	= false,%
		leftline		= false,%
		rightline		= false%
		]{boxclaim}{Claim}[section]
		%comment [section] for global numbering

\newmdtheoremenv [%
		backgroundcolor	= black!10,%
		topline			= false,%
		bottomline	= false,%
		leftline		= false,%
		rightline		= false,%
		linewidth		= 0.5pt%
		]{boxdefinition}{Definition}[section]
		%comment [section] for global numbering

\mdtheorem [%
		backgroundcolor	= black!10,%
		topline			= true,%
		bottomline	= true,%
		leftline		= false,%
		rightline		= false,%
		linewidth		= 0.5pt%
		]{boxcriteria}{Criteria}

% Für Sätze
\mdtheorem [%
		backgroundcolor	= black!10,%
		topline			= true,%
		bottomline	= true,%
		leftline		= false,%
		rightline		= false,%
		linewidth		= 0.5pt%
		]{boxtheorem}{Theorem}

\mdtheorem [%
		backgroundcolor	= black!8,%
		topline			= true,%
		bottomline	= true,%
		leftline		= false,%
		rightline		= false,%
		linewidth		= 0.5pt%
		]{boxlemma}{Lemma}[section]
		%comment [section] for global numbering

\mdtheorem [%
		backgroundcolor	= black!8,%
		topline			= true,%
		bottomline	= true,%
		leftline		= false,%
		rightline		= false,%
		linewidth		= 0.5pt%
		]{boxcorollary}{Corollary}[section]
		%comment [section] for global numbering

\mdtheorem [%
		backgroundcolor	= black!8,%
		topline			= true,%
		bottomline		= true,%
		leftline		= false,%
		rightline		= false,%
		linewidth		= 0.5pt%
		]{boxcollection}{}[section]
		%comment [section] for global numbering

% Number List 
\newlist{numberlist}{enumerate}{1}
\setlist[numberlist, 1]{label={\arabic*.}, itemsep=0em, leftmargin=*,labelindent=0.5em}

% Equation List 
\newlist{eqlist}{enumerate}{1}
\setlist[eqlist, 1]{label={(\roman*)},itemsep=-0.2em, leftmargin=*,labelindent=-0.5em}

% Bullet List 
\newlist{bulletlist}{itemize}{1}
\setlist[bulletlist, 1]{itemsep=0em, leftmargin=0.5em, label={·}}

% Images in multicol 
\newenvironment{Figure}
  {\par\medskip\noindent\minipage{\linewidth}}
  {\endminipage\par\medskip}

\newcommand\tab[1][0.5em]{\hspace*{#1}}

\newcommand{\sectionbreak}{\clearpage}	% Start each section on new page

\usepackage{array}
\newcolumntype{P}[1]{>{\centering\arraybackslash}p{#1}}
\newcolumntype{M}[1]{>{\centering\arraybackslash}m{#1}}

\begin{document}

\begin{multicols}{3}
\setcounter{page}{1}
\pagenumbering{arabic}

\section{Frame Theory}

If $\text{rank}(A)=m$ for $A\in\C^{n\times m}, n\geq m$, then 
$A^\dagger = (A^*A)^{-1}A^*$ is a left-inverse of $A$ and the
solutions to $LA=I$ are 
$$L=A^\dagger+M(I-AA^\dagger), \ \ M\in \C^{n\times m}. $$

\begin{boxdefinition}[Frame]
	Let $H$ be a Hilbert space. A set $(g_k)_k\in H$ is called a frame
    if there exist $0<A\leq B<\infty$ such that
    $$A\|x\|^2\leq \sum_k\|\la x,g_k\ra\|^2\leq B\|x\|^2, 
    \ \ \text{for all } x\in H.$$
\end{boxdefinition}
\begin{boxdefinition}[Analysis \& Synthesis]
    $T\colon H\to \ell^2, \ T(x)=(\la x,g_k\ra)_k$\\
    $T^*\colon \ell^2 \to H, \ T^*(y)=\sum_ky_kg_k$
\end{boxdefinition}

\begin{boxtheorem} [Frame Operator]
    The frame operator $S=T^*T$ satisfies:
    \begin{listnr}
        \item $S^*=S$.
        \item $S$ is positive definite.
        \item $S$ has a square root $S^{1/2}.$
        \item $S$ is invertible.
    \end{listnr}
\end{boxtheorem}

\begin{boxtheorem}[Frame Bounds]
    The tightest possible frame bounds are given by
    the smallest and largest eigenvalues of $S$.
\end{boxtheorem}

\begin{boxtheorem}[Dual Frame]
    For a frame $(g_k)_k$ with frame bounds $A$ and $B$,
    the set $(\tilde{g}_k)_k=(S^{-1}g_k)_k$ is frame with frame bounds 
    $1/B$ and $1/A$, and with analysis operator 
    $\tilde{T}=TS^{-1}$.
     We have $\tilde{T}^*T=T^*\tilde T=\id$.
\end{boxtheorem}

\begin{boxdefinition}[Tight Frame]
    A frame with frame bounds $A=B$ is called a {tight frame.}
\end{boxdefinition}

\begin{boxtheorem}[Tight Frame]
    A frame is tight with frame bound $A$ iff its frame operator is of the
    form $S=A\id.$
\end{boxtheorem}

\begin{boxtheorem}
    For any frame $(g_k)_k$, the set $(S^{-1/2}g_k)_k$ 
    is a tight frame with frame bound 1.
\end{boxtheorem}

\begin{boxtheorem}
    A tight frame $(g_k)_k$ with frame bound $A=1$ and $\|g_k\|=1$
    for all $k$ is an ONB.
\end{boxtheorem}

\begin{boxdefinition}
    A set $(g_k)_k$ is called complete, if 
    $$\la x,g_k\ra =0 \ \forall x\in H \ \imp \ x=0.$$
\end{boxdefinition}

\begin{boxdefinition}
    A frame $(g_k)_k$ is called exact if for all $n$, the set 
    $(g_k)_{k\neq n}$ is incomplete.
\end{boxdefinition}

\begin{boxtheorem} [Exactness]
    Let $(g_k)_k$ be a frame and $(\tilde g_k)_k$ its canonical dual. Then,
    $(g_k)_k$ is exact iff $\la g_j,\tilde g_k\ra=\delta_{jk}.$ 
\end{boxtheorem}

\begin{boxtheorem}[Sampling theorem]
    Let $x\in L^2(\R)$ be bandlimited to $B$, i.e. $\hat x(f)=0$ if 
    $|f|> B$. Then, for $1/T\geq2B$, 
    $$x(t)=2BT\sum_{k=-\infty}^\infty x(kT)\operatorname{sinc}(2B(t-kT)).$$
\end{boxtheorem}

\section{Uncertainty Relations}
\begin{boxdefinition}[Operator Norm]
    $$\triv A=\max_{\|x\|=1}\|Ax\|.$$
\end{boxdefinition}

\begin{boxdefinition}[Frobenius Norm]
    $$\|A\|=\sqrt{\tr (AA^H)}.$$
\end{boxdefinition}

\begin{boxtheorem}
    $$\frac{\| A\|}{\sqrt{\rank A}}\leq\triv A\leq\|A\|.$$
\end{boxtheorem}

\begin{boxdefinition}
    For unitary $U$, we set $$\Delta_{P,Q}(U)=\triv{D_PUD_QU^H}.$$
\end{boxdefinition}

\begin{boxtheorem}
    $$\Delta_{P,Q}(U)=\max_{x\in\im UD_QU^H}\frac{\|D_px\|}{\|x\|}.$$
\end{boxtheorem}

\begin{boxtheorem}
    \begin{align*}
        \frac{\sqrt{\tr(D_PUD_QU^H)}}{\min\{|P|,|Q|\}}\leq\Delta_{P,Q}(U)
        \leq \sqrt{\tr(D_PUD_QU^H)}.        
    \end{align*}
\end{boxtheorem}

\begin{boxdefinition}[Coherence]
    For $A=(a_1\dots a_n)\in \C^{m\times n}$ with normalized columns,
    the coherence is defined as $\mu(A)=\max_{i\neq j}|a_i^Ha_j|$.
\end{boxdefinition}

\begin{boxtheorem}
    $$\Delta_{P,Q}(U)\leq \sqrt{|P||Q|}\mu([I\ U]).$$
\end{boxtheorem}

\begin{boxdefinition}
    Let $P\subseteq \{1,\dots,m\}$ and $\varepsilon_P\in[0,1]$. A
    vector $x\in\C^m$ is called $\varepsilon_P$-concentrated
    if $\|x-x_P\|_2\leq\|\varepsilon_P\|x\|_2.$
\end{boxdefinition}

\begin{boxtheorem}
    Let $A,B\in\C^{m\times m}$ be unitary and $P,Q\subseteq \{1,\dots,m\}$.
    Suppose there exists a nonzero $\varepsilon_P$-concentrated 
    $p\in\C^m$ and a nonzero $\varepsilon_Q$-concentrated $q\in\C^m$
    such that $Ap=Bq.$ Then, 
    $$
    |P||Q|\geq \frac{[1-\varepsilon_P-\varepsilon_Q]_+^2}{\mu([A\ B])^2}
    .$$
\end{boxtheorem}

\begin{boxtheorem}
    Let $A,B\in\C^{m\times m}$ be unitary. If $Ap=Bq$ for nonzero
    $p,q\in\C^m$, then 
    $$
    \|p\|_0\|q\|_0\geq\frac{1}{\mu([A\ B])^2}.
    $$
\end{boxtheorem}

\newpage
\section{Compressive Sensing}

\begin{boxdefinition}[Spark]
    The spark of a matrix is denfined as the cardinality
    of the smallest set linearly dependent columns.
\end{boxdefinition}

\begin{boxtheorem}
    For a matrix $D\in\C^{m\times n}$, uniqueness of 
    recovery of $s$-sparse vectors $x$ from the 
    observation $y=Dx$ is guaranteed if 
    $${\operatorname{spark}(D)}>2s.$$
\end{boxtheorem}

\begin{boxdefinition}[P0]
    $\argmin \|x\|_0$ subject to $y=Dx$.
\end{boxdefinition}

\begin{boxtheorem}
    $$
    1+\frac{1}{\mu(D)}\leq\operatorname{spark}(D)
    $$
\end{boxtheorem}

\begin{boxtheorem}
    If 
    $$
    \|x_0\|_0<\frac{1}{2}\left(1+\frac{1}{\mu(D)}\right),
    $$
    then $x_0$ is the unique solution of (P0).
\end{boxtheorem}

\begin{boxdefinition}[P1]
    $\argmin \|x'\|_1$ subject to $y =Dx'$.
\end{boxdefinition}

\begin{boxtheorem}
    Let $y=Dx_0$ and assume $x_0$ has support set $S$.
    If 
    $$
    \max_{x\in\ker D\setminus\{0\}}
    \frac{\sum_{k\in S}|x_k|}{\sum_k|x_k|}<\frac{1}{2},
    $$
    then $x_0$ is the unique solution of (P1).
\end{boxtheorem}

\begin{boxtheorem}
    If,
    \begin{equation}
    \label{a}
    \|x_0\|_0<\frac{1}{2}\left(1+\frac{1}{\mu(D)}\right),
    \end{equation}
    then 
    $$
    \max_{x\in\ker D\setminus\{0\}}
    \frac{\sum_{k\in S}|x_k|}{\sum_k|x_k|}<\frac{1}{2}
    .$$
    Hence, (\refeq{a}) is a sufficient condition 
    such that $x_0$ is the unique solution of (P1).
\end{boxtheorem}

\begin{boxtheorem}
    If $D\in \C^{m\times n}$, then
    $$
    \mu(D)  \geq \sqrt{\frac{n-m}{m(n-1)}}.
    $$
\end{boxtheorem}

\begin{boxtheorem}
    $$\|x\|_\infty\leq\|x\|_2\leq\|x\|_1
    \leq\sqrt{|\operatorname{supp}x|}\|x\|_2$$
\end{boxtheorem}
\section{Sampling Spectally Sparse Signals}
\begin{boxtheorem}
    Consider a signal with spectral occupancy
    contained in $I$. To reconstruct the signal,
    we need
    $$
    \lim_{r\to\infty}\inf_{t\in\R}\frac{|P\cap [t,t+r]|}
    {r}\geq|I|,
    $$
    where $P=\{t_n\}_n$ denotes the sampling set.
\end{boxtheorem}
\section{Restricted Isometry Property}

\begin{boxdefinition}
    For each $s=1,\dots,n$, the isometry constant $\delta_s$
    of a matrix $\Phi\in\C^{m\times n}$ is the smallest 
    numer such that 
    $$
    (1-\delta_s)\|x\|_2^2\leq\|\Phi x\|_2^2
    \leq(1+\delta_s)\|x\|_2^2
    $$
    for every $s$-sparse $x$.
\end{boxdefinition}

\begin{boxtheorem}
    Let $y=\Phi x$. Assume that $\delta_s<\sqrt{2}-1$.
    Then, the solution $x^*$ to
    $$
    \argmin\|x'\|_1 \ \ \text{subject to } \Phi x'=y
    $$
    satisfies
    $$
    \|x^*-x\|_1\leq C_0\|x-x_s\|_1
    $$
    and
    $$
    \|x^*-x\|_2\leq C_0s^{-1/2}\|x-x_s\|_1
    $$
    for some constant $C_0$. In particular, if $x$ is
    $s$-sparse, recovery is exact.
\end{boxtheorem}

\begin{boxtheorem}
    Let $y=\Phi x+n$. Assume that $\delta_s<\sqrt{2}-1$
    and $\|n\|_2\leq\varepsilon$.
    Then, the solution $x^*$ to
    $$
    \argmin\|x'\|_1 \ \ \text{subject to } 
    \|y-\Phi x'\|\leq\varepsilon
    $$
    satisfies
    $$
    \|x^*-x\|_2\leq C_0s^{-1/2}\|x-x_s\|_1+C_1\varepsilon
    $$
    for some constants $C_0,C1$.
\end{boxtheorem}

\begin{boxdefinition}
    A matrix $\Phi$ is said to satisfy the restricted
    null-space property w.r.t $S$ if 
    $$
    \max_{x\in\ker \Phi\setminus\{0\}}
    \frac{\sum_{k\in S}|x_k|}{\sum_k|x_k|}<\frac{1}{2}.
    $$
\end{boxdefinition}

\begin{boxtheorem}
    If the isometry constant of order $2s$ of $\Phi$
    satisfies $\delta_{2s}<1/3$, then $\Phi$
    satisfies the restricted null-space property
    for any $S$ with $|S|\leq s$.
\end{boxtheorem}
\section{Johnson-Lindenstrauss Lemma}
\begin{boxtheorem}
    Let $\varepsilon\in(0,1)$ and suppose 
    $$
    k\geq \frac{8}{\varepsilon^2-\varepsilon^3}\log(2m)
    .$$
    Then, for every set $X\subset \R^n$ of $m$ points,
    there exists a (linear) map $f\colon \R^n\to \R^k$ such
    that for all $x,x'\in X$ we have
    $$
    (1-\varepsilon)\|x-x'\|^2\leq
    \|f(x)-f(x')\|^2\leq
    (1+\varepsilon)\|x-x'\|^2.
    $$
\end{boxtheorem}


\newpage
\section{Approximation Theory}

Consider a set $C\subseteq L^2(\Omega)$. 

\begin{boxdefinition}[]
    Denote by 
    $$
    \mathfrak{E}^l=\{E\colon C\to\{0,1\}^l\}
    $$
    the set of binary encoders of length $l$
    and by 
    $$
    \mathfrak{D}^l=\{D\colon \{0,1\}^l\to C\}
    $$
    the set of binary decoders of length $l$.
\end{boxdefinition}

\begin{boxdefinition}
    The minimax code length $L(\varepsilon,C)$ for 
    $\varepsilon$ is 
    $$
    L(\varepsilon,C)=\min\big\lbrace l\in\N:\exists (E,D)\in
    \mathfrak{E}^l\times\mathfrak{D}^l:
    \sup_{f\in C}\|D(E(f)) -f \|_2\leq\varepsilon\big\rbrace.$$
    The optimal exponent $\gamma^*(C)$ is defined as
    $$
    \gamma^*(C)=\sup\left\lbrace
    \gamma\in\R:L(\varepsilon,C)\in
    O(\varepsilon^{-1/\gamma}),\ \varepsilon\to 0
    \right\rbrace.$$
\end{boxdefinition}

\begin{boxdefinition}[]
    A metric is a function $d: X\times X\to\R$ that
    satisfies
    \begin{enumerate}
        \item $d(x,y)\geq 0$,
        \item $d(x,y)=0 \Leftrightarrow x=y$,
        \item $d(x,y)=d(y,x)$,
        \item $d(x,z)\leq d(x,y)+d(y,z)$.
    \end{enumerate}
\end{boxdefinition}

\begin{boxdefinition}[$\varepsilon$-covering]
    An $\varepsilon$-covering of a compact set $C$
    with respect to the metric $d$ is a set
    $\{x_1,\dots,x_n\}\subset C$ such that for any
    $x\in C$ there exists an $x_i$ such that 
    $d(x,x_i)\leq \varepsilon$. 
\end{boxdefinition}

\begin{boxdefinition}[Covering Number]
    The $\varepsilon$-covering number 
    $N(\varepsilon, C,d)$ is the cardinality of 
    the smallest $\varepsilon$-covering.
\end{boxdefinition}

\begin{boxdefinition}[Metric Entropy]
    The metric entropy of $C$ is defined 
    as 
    $$
    \log_2 N(\varepsilon,C,d)
    .$$
\end{boxdefinition}

\begin{boxdefinition}[$\varepsilon$-packing]
    An $\varepsilon$-packing of a compact set $C$
    with respect to the metric $d$ is a set 
    $\{x_1,\dots,x_n\}\subset C$ such that 
    $d(x_i,x_j)>\varepsilon$ if $i\neq j$.
\end{boxdefinition}

\begin{boxdefinition}[Packing Number]
    The $\varepsilon$-packing number 
    $M(\varepsilon,C,d)$ is the cardinality of 
    the largest $\varepsilon$-packing.
\end{boxdefinition}

\begin{boxtheorem}[]
    The packing and covering number are related 
    according to 
    $$
    M(2\varepsilon,C,d)\leq N(\varepsilon, C,d)\leq
    M(\varepsilon,C,d)
    .$$
\end{boxtheorem}
\section{Uniform Laws of Large Numbers}

\begin{boxtheorem}[Markov's Inequality]
    Let $X$ be a random variable and assume
    $g:\R\to[0,\infty)$ is increasing. Then,
    for any $c$ with $g(c)>0$, 
    $$
    \P(X\geq c)\leq\frac{\E g(x)}{g(c)}.
    $$
\end{boxtheorem}

\begin{boxtheorem}[Glivenko-Cantelli]
    For any distribution, the empirical CDF $\hat {F}_n$
    satisfies
    $$
    \|\hat F_n-F\|_\infty\to0 \ \ \ \text{a.s.}
    $$
\end{boxtheorem}

\begin{boxdefinition}[]
    Let $\mathcal{F}$ be a set of integrable real-valued
    functions and let $\{X_i\}_{i=1}^n$ be a collection 
    of i.i.d. samples from some distribution $\P$. Then
    we write 
    $$
    \|\P_n-\P\|_\mathcal{F}=\sup_{f\in\mathcal{F}}
    \left|\frac{1}{n}\sum_{i=1}^n f(X_i)-\E f(X)\right|.
    $$
\end{boxdefinition}

\begin{boxdefinition}[Glivenko-Cantelli Class]
    We say that $\mathcal{F}$ is a Glivenko-Cantelli 
    class for $\P$ if $\|\P_n-\P\|_\mathcal{F}$ converges
    to zero in probability as $n\to\infty$.
\end{boxdefinition}

\begin{boxdefinition}[Rademacher Complexity]
    For any fixed collection 
    $x_1^n=(x_1,\dots,x_n)$, consider the set
    $$
    \mathcal{F}(x_1^n)= 
    \left\lbrace(f(x_1),\dots,f(x_n)
    \mid f\in \mathcal{F})\right\rbrace.
    $$
    The empirical Rademacher complexity is defined as 
    $$
    \mathcal{R}(\mathcal{F}(x_1^n)/n)=
    \E_\varepsilon\left[\sup_{f\in\mathcal{F}}
    \left|\frac{1}{n}\sum_{i=1}^n\varepsilon_i
    f(x_i)\right|\right],
    $$
    where $(\varepsilon_i)_{i=1}^n$ is a sequence 
    of Rademacher random variables (uniform on
    $\{-1,+1\}$.)\\
    The Rademacher complexity is defined as
    \begin{align}
        \mathcal{R}_n(\mathcal{F})&=\E_X
        \mathcal{R}(\mathcal{F}(x_1^n)/n)\\
        &=\E_{\varepsilon,X}\left[\sup_{f\in\mathcal{F}}
    \left|\frac{1}{n}\sum_{i=1}^n\varepsilon_i
    f(x_i)\right|\right].
    \end{align}
\end{boxdefinition}

\begin{boxtheorem}[]
    For any $b$-uniformly bounded function class,
    i.e. $\|f\|_\infty\leq b$ for all $f\in\mathcal{F}$
    any $n$ and $\delta\geq0$ we have 
    $$
    \|\P_n-P\|_\mathcal{F}\leq 
    2\mathcal{R}_n(\mathcal{F}) + \delta$$
    with probability at least 
    $1-e^{-\frac{n\delta^2}{2b^2}}.$ Consequently, if
    $\mathcal{R}_n(\mathcal{F})=o(1)$, we have 
    $\|\P_n-\P\|_\mathcal{F}\to0$ a.s.
\end{boxtheorem}

\begin{boxdefinition}[Polynomial Discrimination]
    A class $\mathcal{F}$ of functions has polynomial
    discrimination of order $\nu\geq1$ if for each
    $n$ and $x_1^n=(x_1,\dots,x_n)$ the set 
    $\mathcal{F}(x_1^n)$ has cardinality upper bounded 
    according to 
    $$
    |\mathcal{F}(x_1^n)|\leq (n+1)^\nu.
    $$
\end{boxdefinition}

\begin{boxtheorem}[]
    Suppose that $\mathcal{F}$ has polynomial discrimination
    of order $\nu$. Then, for all $n$ and 
    $x_1^n=(x_1,\dots,x_n)$ we have 
    $$
    \mathcal{R}(\mathcal{F}(x_1^n)/n)
    \leq 4D(x_1^n)\sqrt{\frac{\nu\log(n+1)}{n}},
    $$
    where $D(x_1^n)=\sup_{f\in\mathcal{F}}
    \sqrt{\frac{\sum_{i=1}^nf(x_i)^2}{n}}$.
\end{boxtheorem}

\begin{boxdefinition}[VC Dimension]
    Given a class $\mathcal{F}$
     of binary-valued functions, we
    say that the set
    $x^n_1 = (x_1 , \dots, x_n )$
     is shattered by $\mathcal{F}$
      if $|\mathcal{F}(x^n_1 )| = 2n$.
    The VC dimension $\nu(\mathcal{F})$
    is the largest integer $n$ for which there 
    is some collection $x^n_1 = (x_1 , \dots , x_n )$
     of $n$ points that is shattered by $\mathcal{F}$.
\end{boxdefinition}

\begin{boxtheorem}[Sauer-Shelah]
    Consider a set class $S$ with $\nu(S) < \infty.$
    Then, for any collection of points 
    $x_1^n = (x_1 , \dots , x_n )$ with $n\geq\nu(S)$
    we have 
    $$
    |S(x_1^n)|\leq\sum_{i=0}^{\nu(S)}\binom{n}{i}
    \leq (n+1)^{\nu(S)}.$$

\end{boxtheorem}

\vfill\eject
\columnbreak

\end{multicols}
\end{document}
