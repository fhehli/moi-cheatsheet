\section{Uniform Laws of Large Numbers}

\begin{boxtheorem}[Markov's Inequality]
    Let $X$ be a random variable and assume
    $g:\R\to[0,\infty)$ is increasing. Then,
    for any $c$ with $g(c)>0$, 
    $$
    \P(X\geq c)\leq\frac{\E g(x)}{g(c)}.
    $$
\end{boxtheorem}

\begin{boxtheorem}[Glivenko-Cantelli]
    For any distribution, the empirical CDF $\hat {F}_n$
    satisfies
    $$
    \|\hat F_n-F\|_\infty\to0 \ \ \ \text{a.s.}
    $$
\end{boxtheorem}

\begin{boxdefinition}[]
    Let $\mathcal{F}$ be a set of integrable real-valued
    functions and let $\{X_i\}_{i=1}^n$ be a collection 
    of i.i.d. samples from some distribution $\P$. Then
    we write 
    $$
    \|\P_n-\P\|_\mathcal{F}=\sup_{f\in\mathcal{F}}
    \left|\frac{1}{n}\sum_{i=1}^n f(X_i)-\E f(X)\right|.
    $$
\end{boxdefinition}

\begin{boxdefinition}[Glivenko-Cantelli Class]
    We say that $\mathcal{F}$ is a Glivenko-Cantelli 
    class for $\P$ if $\|\P_n-\P\|_\mathcal{F}$ converges
    to zero in probability as $n\to\infty$.
\end{boxdefinition}

\begin{boxdefinition}[Rademacher Complexity]
    For any fixed collection 
    $x_1^n=(x_1,\dots,x_n)$, consider the set
    $$
    \mathcal{F}(x_1^n)= 
    \left\lbrace(f(x_1),\dots,f(x_n)
    \mid f\in \mathcal{F})\right\rbrace.
    $$
    The empirical Rademacher complexity is defined as 
    $$
    \mathcal{R}(\mathcal{F}(x_1^n)/n)=
    \E_\varepsilon\left[\sup_{f\in\mathcal{F}}
    \left|\frac{1}{n}\sum_{i=1}^n\varepsilon_i
    f(x_i)\right|\right],
    $$
    where $(\varepsilon_i)_{i=1}^n$ is a sequence 
    of Rademacher random variables (uniform on
    $\{-1,+1\}$.)\\
    The Rademacher complexity is defined as
    \begin{align}
        \mathcal{R}_n(\mathcal{F})&=\E_X
        \mathcal{R}(\mathcal{F}(x_1^n)/n)\\
        &=\E_{\varepsilon,X}\left[\sup_{f\in\mathcal{F}}
    \left|\frac{1}{n}\sum_{i=1}^n\varepsilon_i
    f(x_i)\right|\right].
    \end{align}
\end{boxdefinition}

\begin{boxtheorem}[]
    For any $b$-uniformly bounded function class,
    i.e. $\|f\|_\infty\leq b$ for all $f\in\mathcal{F}$
    any $n$ and $\delta\geq0$ we have 
    $$
    \|\P_n-P\|_\mathcal{F}\leq 
    2\mathcal{R}_n(\mathcal{F}) + \delta$$
    with probability at least 
    $1-e^{-\frac{n\delta^2}{2b^2}}.$ Consequently, if
    $\mathcal{R}_n(\mathcal{F})=o(1)$, we have 
    $\|\P_n-\P\|_\mathcal{F}\to0$ a.s.
\end{boxtheorem}

\begin{boxdefinition}[Polynomial Discrimination]
    A class $\mathcal{F}$ of functions has polynomial
    discrimination of order $\nu\geq1$ if for each
    $n$ and $x_1^n=(x_1,\dots,x_n)$ the set 
    $\mathcal{F}(x_1^n)$ has cardinality upper bounded 
    according to 
    $$
    |\mathcal{F}(x_1^n)|\leq (n+1)^\nu.
    $$
\end{boxdefinition}

\begin{boxtheorem}[]
    Suppose that $\mathcal{F}$ has polynomial discrimination
    of order $\nu$. Then, for all $n$ and 
    $x_1^n=(x_1,\dots,x_n)$ we have 
    $$
    \mathcal{R}(\mathcal{F}(x_1^n)/n)
    \leq 4D(x_1^n)\sqrt{\frac{\nu\log(n+1)}{n}},
    $$
    where $D(x_1^n)=\sup_{f\in\mathcal{F}}
    \sqrt{\frac{\sum_{i=1}^nf(x_i)^2}{n}}$.
\end{boxtheorem}

\begin{boxdefinition}[VC Dimension]
    Given a class $\mathcal{F}$
     of binary-valued functions, we
    say that the set
    $x^n_1 = (x_1 , \dots, x_n )$
     is shattered by $\mathcal{F}$
      if $|\mathcal{F}(x^n_1 )| = 2n$.
    The VC dimension $\nu(\mathcal{F})$
    is the largest integer $n$ for which there 
    is some collection $x^n_1 = (x_1 , \dots , x_n )$
     of $n$ points that is shattered by $\mathcal{F}$.
\end{boxdefinition}

\begin{boxtheorem}[Sauer-Shelah]
    Consider a set class $S$ with $\nu(S) < \infty.$
    Then, for any collection of points 
    $x_1^n = (x_1 , \dots , x_n )$ with $n\geq\nu(S)$
    we have 
    $$
    |S(x_1^n)|\leq\sum_{i=0}^{\nu(S)}\binom{n}{i}
    \leq (n+1)^{\nu(S)}.$$

\end{boxtheorem}